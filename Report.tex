\documentclass{article}
\usepackage[table,xcdraw]{xcolor}
\usepackage[utf8]{inputenc}
\usepackage{geometry}
\usepackage{amsmath}
\usepackage{amsthm}
\usepackage{amsfonts}
\usepackage{amssymb}
\usepackage{graphicx}
\usepackage{tocloft}
\usepackage{pgfplots}
\usepackage{multirow}	
\usepackage{tikz}

\usepackage[usenames,dvipsnames]{pstricks}
\usepackage{epsfig}
\usepackage{pst-grad} % For gradients
\usepackage{pst-plot} % For axes
\usepackage[space]{grffile} % For spaces in paths
\usepackage{etoolbox} % For spaces in paths

%\usepackage{perpage} %the perpage package
%\MakePerPage{footnote} %the perpage package command

% For adding TOC in pdf bookmarks
\usepackage{hyperref}
\hypersetup{
	colorlinks=true,
	linkcolor=blue,
	filecolor=magenta,      
	urlcolor=cyan,
}
\usepackage{hypcap}
%
\usepackage{float}
\usepackage{listings}
\usepackage{color}

\definecolor{codegreen}{rgb}{0,0.6,0}
\definecolor{codegray}{rgb}{0.5,0.5,0.5}
\definecolor{codepurple}{rgb}{0.58,0,0.82}
\definecolor{backcolour}{rgb}{0.95,0.95,0.92}

\lstdefinestyle{mystyle}{
	backgroundcolor=\color{backcolour},   
	commentstyle=\color{codegreen},
	keywordstyle=\color{magenta},
	numberstyle=\tiny\color{codegray},
	stringstyle=\color{codepurple},
	basicstyle=\footnotesize,
	breakatwhitespace=false,         
	breaklines=true,                 
	captionpos=b,                    
	keepspaces=true,                 
	numbers=left,                    
	numbersep=5pt,                  
	showspaces=false,                
	showstringspaces=false,
	showtabs=false,                  
	tabsize=2
}

\lstset{style=mystyle}
\usepackage{xepersian}
\usepackage{bidi}

\settextfont{Yas}
\SepMark{-}

\renewcommand{\cftsecleader}{\cftdotfill{\cftdotsep}}

\theoremstyle{definition}
\newtheorem{definition}{تعریف}

\title{تمرین درس طراحی کامپایلرها}
\author{امیر حقیقتی ملکی}
\date{پاییز ۹۶}
	
\begin{document}
	%%%%%%%%%%%%%%%%%%%%%%%%%%%%%%%
	%%	 TITLE PAGE - BEGIN	     %%
	%%%%%%%%%%%%%%%%%%%%%%%%%%%%%%%
	\newgeometry{margin=1in}
	\pagenumbering{gobble}
		\begin{titlepage}
		\centering
		\includegraphics[width=0.25\textwidth]{../../../Template/Resources/logo.png}\par\vspace{1cm}
		{\scshape\LARGE دانشگاه صنعتی امیرکبیر \par}
		{\scshape\LARGE دانشکده مهندسی کامپیوتر و فناوری اطلاعات \par}
		\vspace{1cm}
		{\scshape\Large
	گزارش پروژه
			\par}
		\vspace{1.5cm}
		{\huge\bfseries 
			گزارش پروژه سوم
			\par}
		\vspace{2cm}
		{\Large امیر حقیقتی \par}
		{\Large ۹۳۳۱۰۰۹\par}
		\vfill
		استاد درس:\par
		دکتر ممتازی
		\vfill
		
		% Bottom of the page
		{\large \rl{
				زمستان ۹۶
			}\par}
	\end{titlepage}
	\newpage
	\pagenumbering{gobble}
	\tableofcontents
	\newpage
	\pagenumbering{arabic}
	\section{مقدمه}
	پروژه اول و دوم درس طراحی کامپایلرها که در ترم اول سال تحصیلی ۹۷-۹۶ ارائه می‌گردد، با هدف پیاده‌سازی یک تحلیل‌گر واژگان\footnote{Lexer} و همچنین یک تحلیل‌گر نحوی\footnote{Parser} برای یک گرامر مفروض است. به منظور پیاده‌سازی این دو مورد،  از زبان JAVA استفاده شد که طی فاز اول آن فقط با استفاده از ابزار  JFlex توانستم تحلیل‌گر واژگان را پیاده‌سازی کنم که در فاز دوم صرفا عملیات در صورت برخورد به هر کلیدواژه را عوض نموده و به جای نوشتن در خروجی، به تحلیل‌گر نحوی پاس می‌دهد. \\
	همچنین مطابق تعریف پروژه، از ابزار BISON برای تولید تحلیل‌گر نحوی استفاده شد که در ادامه جزییات آن ذکر می‌شود.
	\section{شرح پیاده‌سازی}
	\subsection{فاز ۱ (تحلیل‌گر واژگان)}
	با توجه به گرامر داده شده، ابتدا تمامی کلیدواژه‌ها و حروف رزرو استخراج شدند که برای نوشتن فایل توصیف JFlex مورد نیاز بودند. سپس با تعریف مناسب آن‌ها در قالب فایل توصیف JFlex، عملیات مورد نیاز در صورت مشاهده هر کلیدواژه تعریف شد که در این فاز صرفا نوشتن آن کلید واژه به همراه ضمایم و متعلقاتش در کنسول خروجی بود. فایل توصیف JFlex فاز اول،‌ در آدرس زیر موجود و قابل مشاهده است:
	\begin{latin}
\begin{lstlisting}[language=bash]
src\Ph1\lexer.flex
\end{lstlisting}
	\end{latin}
	\subsection{فاز ۲ (تحلیل‌گر نحوی)}
	\section{مسائل و مشکلات}
\end{document}